\chapter{Einleitung}
In meiner Arbeit über Ethik und Daten geht es um die Gefahren welche eine KI mit sich brignt. Da dieses Thema jedoch so gigantisch ist, dass ich mehr als nur 1 Leben dieser Arbeit widmen könnte, werde ich hauptsächlich über die Gefahren sprechen von welchen viele Menschen angst haben.
Um ein besseres Verständnis zu erlangen werde ich Ihnen jedoch zuerst die allgemeinen Funktionen von KI nennen.
\\
\\
\section{Was ist KI?}
Eine simple und eindeutige Definition für den Begriff KI gibt es bisher noch nicht. Dies liegt jedoch auch daran, dass noch nicht einmal der Begriff Intelligenz definiert ist.
Viele Forscher sprechen auch nicht von KI sondern viel mehr von komplexen Algorithmen. Man kann jedoch KI nicht einfach als Algorithmus bezeichnen, denn bei einer KI kommt es auch auf die Rechenleistung oder die Qualität der Daten an mit welcher sie trainiert wird. 
Fakt ist also, dass Algorithmen anhand von Beispielen lernen, Aufgaben eigenständig auszuführen, indem sie in einer für Menschen oft unüberschaubaren Fülle an Daten Muster erkennen. Unter anderem aus diesem zielgerichteten Datenlernen resultiert bei KI der Intelligenzbegriff.
Künstliche Intelligenz viel eher ein Sammelbegriff für: Algorithmen, Trainings- und Lernprozesse und die Datenauswahl. Aus dieser Kombination entstehen Programme welche sich selber verbseern und teilweise auch Code generieren können.
\\
\\
\section{Wie lernt KI?}
KI hat verschiedene Möglichkeiten um zu lernen. Die wichtigsten sind jedoch:
\begin{itemize}
    \item Maschinelles Lernen
    \item Deeplearning
\end{itemize}

\subsection{Maschinelles Lernen}

Mit dem Maschinellen Lernen lernt eine KI wie ein Kind welches durch Erfahrung lernt. Sie lernen dabei mithilfe von Variablen welche die Beziehungen zwischen Variablen (d. h. Muster) entdecken und dann aus diesen Lektionen lernen.

\subsection{Deeplearning}

Deeplearning hat viele Ähnlichkeiten zu dem menschlichen Gehirn und wie dieses Daten verarbeitet.
Es bildet aus dem eigenen Datensatz Neuronale Netze, welche sich dann immer weiter verknüpfen, verzweigen und von sich von selbst immer weitere Verbindungen eingehen. Aus diesen Netzen kann der Algorithmus dann von selbst Schlüsse ziehen um Probleme zu lösen.
Dabei ist der grösste Unterschied zum menschlichen Gehirn, dass das Gehirn nur wenige Daten braucht um z.B. zwischen einem Hund und einer Katze benötigt. Ein Algorithmus braucht dafür jedoch extrem viele Daten,
 dafür ist der Algorithmus speziell auf schwereren Fragen, welche sehr explizit sind, genauer und auch schneller.
\\
Zusammenfassend lässt sich sagen, dass KI eine autodidaktisches System ist, welches von seinem Ersteller eingeschränkt werden kann.  \citep{Was-ist-KI?}

\subsection{KI im Alltag}
KI wird mitlerweile im Alltag schon fast überall verwendet. Sie wegzudenken ist schon fast unmöglich, denn Sie steck in jedem sich denkbaren Gerät von Smartphones über zu Autos bis hin zu Kühlschränken.
Es gibt unglaublich viele Möglichkeiten was wir nun mit dieser neuen Macht anstellen. Die Welt wird zu 100\% dadurch verändert. Dies kann jedoch auch im negativen passieren und da kommen wir zu meinen Fragen welche ich mir für diese Arbeit gestellt habe.


\section{Meine Fragen}
Ich habe mich schon früh als kleines Kind für den Themenbereich Informatik und da ist nur naheliegend, dass ich schon früh auf KI und dessen Gefahren gestossen bin.
Anfangs waren alles noch hauptsächlich Spekulationen, aber nun fast 10 Jahre später ist man diesem Thema um einiges näher und dennoch habe ich immer weniger Angst davor.
Ich habe mich daher gefragt ob es gut ist weniger Angst davor zu haben oder ob ich hier eine Blindheit entwickle was sich in schon naher Zukunft auswirken könnte.
Daher habe ich mir folgdene Frage gestellt: \textbf{"Wird KI für den Menschen gefährlich?"}
\\
Da diese Frage GIGANTISCH ist habe ich sie in folgende 3 Unterfragen aufgeteilt:
\begin{itemize}
    \item Verselbstständigt sich KI?
    \item KI ilegal nutzen?
    \item Bedroht KI Jobs?
\end{itemize}