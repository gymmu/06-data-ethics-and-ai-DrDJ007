\chapter{Bedroht KI Jobs?}
Ja, in der Schweiz sind schon 60\% der Jobs von KI beinflusst. Also um einiges mehr als die Hälfte vermutlich auch mehr als Sie erwartet haben. Besonders sind jedoch folgende Jobs und Berufsgruppen:
\begin{itemize}
    \item Buchhalter/-innen
    \item Mathematiker/-innen
    \item Programmierer/-innen
    \item Kund(/-innen)support
    \item Büro und Administartion
    \item Transportwesen
    \item Zusteller/-innen
    \item Lebensmittelproduktion
\end{itemize}
Jedoch werden auch neue Jobs geschaffen werden, man teilt die neuen Jobs in folgende 3 Kategorien auf:
\begin{itemize}
    \item Trainer/-innen
    \item Erhalter/-innen
    \item Erklärer/-innen
\end{itemize}
Dabei sind Trainer/-innen hauptsächlich für die Entwicklung und Trainierung der Software zuständig. Sie müssen also schauen, dass alles funktioniert wie es soll.
Erhalter/-innen sollen für das Gleichgewicht zwischen der Einhaltung der finanziellen Zielen und der Performance der KI zu überwachen und auch zu erhalten.
Und was müssen die Erklärer/-innen machen? Die Erklärer/-innen müssen die Entscheidungen der KI's einordnen und bewerten.\citep{bedrohte-jobs-kununu}
\\
Ich selbst bin hier ein wenig unsicher was ich von der ganzen Sache halten soll. Zum einen sehe ich den  gewaltigen Fortschritt welcher KI uns bringt, zum anderen habe ich jedoch immernoch das Gefühl,
dass KI uns etwas weg nimmt und ich meine nicht nur Jobs. Ich finde, dass es uns so langsam das ganze Denken abnimmt und so wirklich befürworten kann ich dies nicht. Abgesehen davon denke ich das KI weniger Jobs schafft als das sie wegnimmt.
\\
\\
\section{Fazit}
Ich bleibe nach wie vor, dass KI eingeschränkt benutztbar sein soll. Aufjeden Fall nicht so offen wie es jetzt ist. Auch sollte man sich vielleicht ein paar Gedanken darüber machen, speziell als junge Person, wie man sich denn die nahe Zukunft aussieht, denn es wird fast sicher anderst als wir denken.
