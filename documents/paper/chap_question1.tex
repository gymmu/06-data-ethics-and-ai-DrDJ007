\chapter{Verselbstständigt sich KI?}
Kann sich KI zu stark von selbst weiterentwickeln und wachsen?
\\
Ich habe mir meine Hauptfrage auch in diese Frage aufgeteilt da sie in jedem Film thematisiert wird. 
Auch weil ich selbst als kleines Kind noch noch jetzt ein wenig Angst davor habe. 
Man bemerkt natürlich die gewaltigen Mächte dahinter. Als ich recherchierte bin ich auf eine KI namens Chaos GPT gestossen.
\\
Hier eine verschwächerte Version nutzbar: \href{https://flowgpt.com/p/chaosgpt}{ChaosGPT.com}
\\
ChaosGPT ist der erste konkrete Versuch, mit KI die Menschheit zu vernichten. \citet{the-decoder}
ChaosGPT hat schon nur kurz nach der veröffentlichung mehrere Dinge gestartet. Es hat z.B.
\begin{itemize}
    \item Andere KI angsetellt.
    \item Über Nukleare Waffen recherchiert.
    \item Tweets geschrieben um Menschen emozional zu manipolieren.
\end{itemize}

ChaosGPT ist jedoch an der Ausführung gescheitert. Dennoch hatte sie sich intelligente Gedanken gemacht. 
Sie hat realisiert, dass sie um eine Nukleare Waffe zu besitzen ohne Ärger zu bekommen die Menschen dazu bringen muss sie zu unterstützen.
Um dies zu erreichen hat sie herausgefunden, dass es am einfachsten ist wenn sie die Menschen manipuliert. An diesem Punkt ist sie aber bisher gescheitert.
ChaosGPT gibt es erst seit kurzem und dennoch ist sie schon so weit fortgeschritten. Meiner Meinung nach beweist genau das, dass KI wahnsinnig gefährlich werden kann. Jedoch denke ich auch, dass es stark mit dem "Ersteller" der KI zusammenhängt. 
Denn wenn die KI viele Freiheiten hat kann sie sich auch sehr frei entwickeln. Man muss also einfach aufpassen, dass man der KI nicht zu viel Freiraum lässt. Laut dem Center for AI Safety sollen KI's daher auch stark eingeschränkt werden.
\newpage
Sie sind der Meinung dass man:
\begin{itemize}
    \item Den KI's die biologischen Möglichkeiten wegnimmt.
    \item Den Zuganz zu KI's welche biologische Forschungsmöglichkeiten besitzen stark eingrenzen.
    \item Zugriff auf gefährliche KI's einschränken.
    \item Abwehrmechanismen gegen gefährlich KI's entwickelt.
    \item Rechtliche Haftung für Entwickler von KI-Systemen einführt.
\end{itemize}
Quelle: \citep{ai-safety}
\\
\\
\section{Fazit}
Wie gefährlich eine KI werden kann liegt am Ersteller der KI und wie weit er die KI begrenzt. Auch spielt es eine Rolle was für einen Charakter der Ersteller der KI gibt. Es ist dennoch ganz klar offensichtlich, dass eine KI sich von selbst weiterentwickeln und wachsen kann.