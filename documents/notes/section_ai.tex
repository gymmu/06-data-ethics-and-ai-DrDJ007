\section{Künstliche Intelligenz}
\label{sec:ai}

In diesem Abschnitt sind meine Notizen zu künstlicher Intelligenz zu finden.

Künstliche Intelligenz ist ein Teilgebiet der Informatik und beschäftigt sich mit maschinellem Lernen \citep{ai-wikipedia}.
\section{Meine Fragen}
Kann KI für den menschen "gefährlich" werden?
\\
Ich habe diese Frage in drei weitere Fragen aufgeteilt.
\newline Nämlich: 
\begin{itemize}
    \item Kann sich KI zu stark von selbst weiterentwickeln und wachsen?
    \item KI ilegal nutzen?
    \item Wird KI den Menschen die Jobs klauen?
\end{itemize}
\section{Kann sich KI zu stark von selbst weiterentwickeln und wachsen?}
Kann sich KI zu stark von selbst weiterentwickeln und wachsen?
\\
Ich habe mir meine Hauptfrage auch in diese Frage aufgeteilt da sie in jedem Film thematisiert wird. 
Auch weil ich selbst als kleines Kind noch noch jetzt ein wenig Angst davor habe. 
Man bemerkt natürlich die gewaltigen Mächte dahinter. Als ich recherchierte bin ich auf eine KI namens Chaos GPT gestossen.
\\
Hier eine verschwächerte Version nutzbar: \href{https://flowgpt.com/p/chaosgpt}{ChaosGPT.com}
\\
ChaosGPT ist der erste konkrete Versuch, mit KI die Menschheit zu vernichten. \citep{the-decoder}
ChaosGPT hat schon nur kurz nach der veröffentlichung mehrere Dinge gestartet. Es hat z.B.
\begin{itemize}
    \item Andere KI angsetellt.
    \item Über Nukleare Waffen recherchiert.
    \item Tweets geschrieben um Menschen emozional zu manipolieren.
\end{itemize}

ChaosGPT ist jedoch an der Ausführung gescheitert. Dennoch hatte sie sich intelligente Gedanken gemacht. 
Sie hat realisiert, dass sie um eine Nukleare Waffe zu besitzen ohne Ärger zu bekommen die Menschen dazu bringen muss sie zu unterstützen.
Um dies zu erreichen hat sie herausgefunden, dass es am einfachsten ist wenn sie die Menschen manipuliert. An diesem Punkt ist sie aber bisher gescheitert.
ChaosGPT gibt es erst seit kurzem und dennoch ist sie schon so weit fortgeschritten. Meiner Meinung nach beweist genau das, dass KI wahnsinnig gefährlich werden kann. Jedoch denke ich auch, dass es stark mit dem "Ersteller" der KI zusammenhängt. 
Denn wenn die KI viele Freiheiten hat kann sie sich auch sehr frei entwickeln. Man muss also einfach aufpassen, dass man der KI nicht zu viel Freiraum lässt. Laut dem Center for AI Safety sollen KI's daher auch stark eingeschränkt werden.
\clearpage
Sie sind der Meinung dass man:
\begin{itemize}
    \item Den KI's die biologischen Möglichkeiten wegnimmt.
    \item Den Zuganz zu KI's welche biologische Forschungsmöglichkeiten besitzen stark eingrenzen.
    \item Zugriff auf gefährliche KI's einschränken.
    \item Abwehrmechanismen gegen gefährlich KI's entwickelt.
    \item Rechtliche Haftung für Entwickler von KI-Systemen einführt.
\end{itemize}
Quelle: \citep{ai-safety}
\\
\\
{\large Fazit (von Frage 1):}
\\
Wie gefährlich eine KI werden kann liegt am Ersteller der KI und wie weit er die KI begrenzt. Auch spielt es eine Rolle was für einen Charakter der Ersteller der KI gibt. Es ist dennoch ganz klar offensichtlich, dass eine KI sich von selbst weiterentwickeln und wachsen kann.
\\
\section{KI ilegal nutzen?}
Zuerst dachte ich, dass es gar nicht so viele Optionen geben kann KI ilegal zu nutzen, denn es ist doch hauptsächlich ein Hilfsmittel.
Später kam mir dann der Gedanke wie es denn mit Gesichtserkennung und ständiger überwachung aussieht. Dies sollte jedoch gar nicht wirklich ilegal sein. 
Ich lag jedoch auf der richtigen Spur, denn das ilegalste was man mit KI tun kann ist: Deepfake.
Deepfakes sind realistisch wirkende Medieninhalte, die durch Techniken der künstlichen Intelligenz abgeändert, erzeugt, also verfälscht worden sind.\citep{deepfake-wikipedia}
Deepfake wird fast überall angewendet. Jedoch hauptsächlich bei:
\begin{itemize}
    \item Politik
    \item Kunst
    \item Datenschutz
    \item Forschung
    \item Pornografie
\end{itemize}
Diese neue Kunst der Manipulation bringt natürlich gewaltige Probleme mit sich, z.B. kann man nun eigentlich keinen Medien mehr glauben schenken, oder sobald man sein Gesicht irgendwo gepostet hat kann es für Pornografie oder anderes ausgenutzt werden.
Speziell schlimm finde ich jedoch gefälschte Anrufe, also wenn man anhand einer auch nur noch so kurzen Sprachdatei eine eigene KI trainiert, welche genau so spricht wie diese Person. Dies ist auch eine grosse Betrugsmasche auf welche besonders ätere Personen reinfallen. Man muss jedoch aufpassen, denn wenn die ganzen KI's sich mit einer solchen geschwindigkeit weiterentwickeln können selbst gut informierte bald nicht mehr zwischen echt und generiert unterscheiden.
Unter folgendem Link könnt ihr eure eigen Stimmbot trainieren oder Stimmen von Olaf Scholz, Angela Merkel, usw. verwenden \href{https://de.vidnoz.com/stimme-klonen.html?insur=degooglecamp_voiceclone_stimmen%20ai%20eigene%20stimme&gad_source=1&gclid=CjwKCAjwx-CyBhAqEiwAeOcTdXzrCxZKDQlEb6uY7WpuKmMl4A4Kjq1nL8A8dgPwKI11yVB6GCWjixoCfvIQAvD_BwE}{vidnoz.com}.
Was momentan noch wegen witzig verwendet wird kann auch ganz anderst verwendet werden und es ist nicht einmal schwer.
